
% !TeX encoding = UTF-8


\documentclass[12pt,a4paper]{article}
\usepackage[utf8]{inputenc}
\usepackage[spanish]{babel}
\usepackage{amsmath, amssymb}
\usepackage{graphicx}
\usepackage{booktabs}
\usepackage{hyperref}
\usepackage{geometry}
\geometry{left=3cm,right=3cm,top=3cm,bottom=3cm}


\title{Simulación del comportamiento de votantes en una elección local\\
       \large Modelo de ecuaciones diferenciales con escenarios de propaganda y redes sociales}
\author{Nombre del autor \\ \texttt{correo@ejemplo.com}}
\date{\today}

\begin{document}
\maketitle
\begin{abstract}
Se desarrolla un modelo determinista basado en ecuaciones diferenciales para estudiar la dinámica de intención de voto de la población ante estrategias de campaña tradicionales (medios masivos) y campañas dominadas por redes sociales.
El sistema incluye tres candidatos (A, B, C), un compartimento de indecisos y uno de abstención.
Se calibra un conjunto de parámetros hipotéticos y se analizan tres escenarios: \emph{Medios}, \emph{Redes} y \emph{Mixto}.
Se presentan resultados de series temporales, un mapa de calor de sensibilidad y un análisis de número de persuasión básica.
\end{abstract}

\section{Introducción}
La literatura sobre difusión de opiniones utiliza tanto modelos basados en agentes (ABM) como aproximaciones agregadas (ODE).
Inspirados en los modelos compartimentales epidemiológicos (SEIR), adoptamos la segunda aproximación por su transparencia analítica y facilidad para describir cada parámetro.
Nuestro objetivo es cuantificar cómo cambia la distribución de preferencias electorales cuando se modifican dos palancas principales: \textbf{propaganda masiva} ($\beta$) y \textbf{influencia social} ($\lambda$).

\section{Materiales y métodos}

\subsection{Compartimentos}
\begin{itemize}
   \item $U(t)$: indecisos susceptibles de persuasión.
   \item $A(t)$, $B(t)$, $C(t)$: simpatizantes de cada candidato.
   \item $N(t)$: abstención acumulada.
   \item Población total normalizada: $U + A + B + C + N = 1$.
\end{itemize}

\subsection{Sistema de ecuaciones}
\[
\begin{aligned}
\dot U &= -(\beta_A+\lambda_A A)U-(\beta_B+\lambda_B B)U-(\beta_C+\lambda_C C)U \\
       &\qquad -\alpha U + \mu(A+B+C) \\[4pt]
\dot A &= (\beta_A+\lambda_A A)U-\mu A-\sigma A(B+C) \\[2pt]
\dot B &= (\beta_B+\lambda_B B)U-\mu B-\sigma B(A+C) \\[2pt]
\dot C &= (\beta_C+\lambda_C C)U-\mu C-\sigma C(A+B) \\[2pt]
\dot N &= \alpha U
\end{aligned}
\]

Cada término de entrada/salida se justifica como transferencia de personas entre estados o pérdida de interés (\S\ref{sec:param}).

\subsection{Parámetros}
\label{sec:param}
\begin{tabular}{lllrrr}
\toprule
    Símbolo &             Descripción & Unidad &  Medios &  Redes &  Mixto \\
\midrule
  $\beta_A$ & Propaganda masiva pro‑A &  1/día &   0.040 &  0.005 &  0.020 \\
  $\beta_B$ & Propaganda masiva pro‑B &  1/día &   0.015 &  0.005 &  0.015 \\
  $\beta_C$ & Propaganda masiva pro‑C &  1/día &   0.005 &  0.005 &  0.008 \\
$\lambda_A$ &      Influ. social de A &  1/día &   0.200 &  1.000 &  0.600 \\
$\lambda_B$ &      Influ. social de B &  1/día &   0.200 &  1.000 &  0.600 \\
$\lambda_C$ &      Influ. social de C &  1/día &   0.100 &  1.000 &  0.400 \\
      $\mu$ &           Fatiga/olvido &  1/día &   0.030 &  0.030 &  0.030 \\
   $\alpha$ &      Tasa de abstención &  1/día &   0.010 &  0.010 &  0.010 \\
   $\sigma$ &       Contra‑persuasión &  1/día &   0.000 &  0.000 &  0.010 \\
\bottomrule
\end{tabular}


\subsection{Escenarios}
\begin{description}
  \item[\textbf{Medios}] $\beta$ altos,  $\lambda$ bajos.
  \item[\textbf{Redes}] $\beta$ bajos,  $\lambda$ altos.
  \item[\textbf{Mixto}] valores intermedios y $\sigma>0$.
\end{description}

\subsection{Implementación}
El sistema se resuelve con \texttt{scipy.integrate.solve\_ivp}.
El código está disponible en el paquete \texttt{src/} del repositorio Git, bajo licencia MIT.

\section{Resultados}

\subsection{Dinámica temporal}
\begin{figure}[ht]
   \centering
   \includegraphics[width=\textwidth]{figures/compare_ABCN_Medios.png}
   \caption{Evolución temporal de cada compartimento en el escenario \emph{Medios}.}
\end{figure}

\subsection{Sensibilidad del modelo}
\begin{figure}[ht]
   \centering
   \includegraphics[width=.8\textwidth]{figures/heatmap_beta_lambda_A.png}
   \caption{Mapa de calor del voto final de A en función de $\beta_A$ y $\lambda_A$.}
\end{figure}

\subsection{Métricas resumidas}
\begin{itemize}
  \item Voto final A/B/C/N por escenario.
  \item Número de persuasión básica $R_P = (\beta+\lambda)/\mu$.
  \item Área bajo la curva (AUC) como medida de apoyo acumulado.
\end{itemize}

\section{Discusión}
Los resultados muestran que bajo fuerte propaganda tradicional, el candidato con mayor presupuesto ($A$) domina rápidamente, mientras que un escenario de redes produce distribuciones más equilibradas y depende de las condiciones iniciales.
El compartimento de abstención $N$ crece si $\alpha>0$ y limita el voto efectivo.

\subsection*{Limitaciones}
\begin{enumerate}
  \item Suposiciones homogéneas: no se modela la estructura real de la red.
  \item Parámetros hipotéticos: falta calibración con datos de encuestas.
  \item No se incluye retroalimentación negativa por saturación mediática.
\end{enumerate}

\section{Conclusiones}
El modelo ODE, aunque simplificado, permite comparar cuantitativamente estrategias de campaña y extraer umbrales de efectividad.
Valores de $R_P>1$ indican crecimiento auto-sostenido de simpatizantes; la combinación $\beta$–$\lambda$ necesaria depende fuertemente de la tasa de olvido $\mu$.

\section*{Código y datos}
Repositorio: \url{https://github.com/tu_usuario/proyecto_simulacion}.
Scripts para reproducir figuras: \texttt{src/experiments/\*.py}.
\end{document}
